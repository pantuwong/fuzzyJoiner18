\subsection{Approximate Nearest Neighbors}
We used the ANNOY package to find the nearest neighbors in output of the hidden layer computed in the previous section.\cite{annoy_impl} This package approximates the nearest neighbors by first calculating binary trees using the randomized k-d forest method.\cite{ann_paper} These trees are calculated by splitting the vector space by drawing random hyperplanes. Each side of the hyperplane is one node of the binary tree. The algorithm then continues to split each subspace recursively until no more than a predetermined number of items are in each subsection. Some of the neighbors can be found by looking at the items in the same subspace, or subspaces near the target item in the tree. The processes of building a tree is repeated a number of times. By taking a union of the points from all the trees, a reasonable neighborhood is approximated. Once we have a neighborhood of points, we can calculate distance and sort the subset of points. Since much of the time goes into creating the trees, queries can be executed quickly, in logarithmic time.
